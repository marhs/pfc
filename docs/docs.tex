\documentclass[10pt, a4paper, spanish]{report}
\usepackage[spanish]{babel}
\selectlanguage{spanish}
\usepackage[utf8]{inputenc}
\usepackage{amsmath}
% TODO Configurar el documento para que use codigo python
\usepackage{listings}


\begin{document}

\title{Descripción y analisis de un protocolo autenticado de transferencia de claves en grupo basado en la comparticion de secretos.}
\author{Marco Herrero Serna,\\
        Escuela Tecnica Superior de Ingeniería Informatica,\\
        Universidad de Sevilla\\
        \texttt{me@marhs.de}}
\date{\today}
\maketitle


% TODO Rellenar el abstract
\begin{abstract}
Un estudio y posterior implementación de un protocolo de transferencia de claves en grupo. Está basado en usuarios autenticados a un Key Generatino Center que las distribuye mediante tecnicas de compartición de secretos. Se estudia tambien las debilidades de este metodo, las mejores que ofrece sobre otros diferetes y el state of the art. 
\end{abstract}

\section{Introducción}
 

En la epoca actual, las oportunidades de comunicacion están creciendo exponencialmente gracias a las nuevas tecnologías. Ahora se genera mas información que en los ultimos 10 años (o algo así, falta cita). Esto hace necesario un aumento del desarrollo de las tecnologías de la información (IT) y de sus capacidades para asegurar esta información. \\

La seguridad en TI siempre ha aprovechado la capacidad de la criptografia para proveer de seguridad en los diferentes ambientes necesarios. Grandes ejemplos de esto son el uso de la criptografia de clave publica desde la invencion de RSA (cita), pasando por los algoritmos resumen. Una falla en estos sistemas matematicos resulta en una perdida de seguridad para las tecnologias de la informacion. \\

    \subsection{La comunicacion en grupo}
    Uno de los aspectos de los que se encarga la criptografia es aportar seguridad (integridad, confidencialidad y trazabilidad?) a las comunicaciones en un grupo de participantes, mas alla de dos. Esto puede hacerse mediante un participante central que se encargue de recibir y transmitir los mensajes de todos los participantes en el grupo, organizando la seguridad como una comunicacion entre cada participante y el participante central. La otra manera es dejar que los participantes se comuniquen entre ellos sin tener que enviar sus mensajes a traves de un tercero. Para esto, requieren un secreto comun solo conocido por estos.

\section{Descripcion del problema}
La seguridad de la comunicacion grupo, como se ha visto, esta basada en un tercer elemento que aporte la seguridad a la conversacion. Este tercer elemento puede ser un administrador externo que difunda los mensajes manteniendo la seguridad o un sistema criptografico conocido entre todos los participantes. Este documento va a centrarse en este ultimo caso.

    \subsection{Clave simetrica vs clave publica}
    Cuando se habla de un sistema criptografico para la seguridad de un mensaje existen dos sistemas generales a tener en cuenta. El sistema de clave publica/privada, donde el emisor no comparte una clave con el receptor, es una forma de garantizar la seguridad de una comunicacion sin tener que difundir la clave, pero 
    \subsection{Acuerdo de clave}
    \subsection{Casos prácticos o ejemplos }
    \subsection{Soluciones externas }
    \subsection{Comparativas }

\section{Solucion presentada}
Descripcion y limitaciones

\section{Análisis de requisitos y planificación}
    \subsection{Planificacion }
    Búsqueda de problemas
    Ataque de problemas
    Apartado practico, etc

\section{Implementación}
Interfaz grafica, kivy pyinstaller, etc. 

Uno de los problemas planteados para la implementación es la solución de la ecuación

\begin{equation}
    K' = g^{s_{i}'} * g^{s_{i}+r_{i}} / g^{r_{i}}
\end{equation}

dónde al conocer los valores ya calculados de $ g^{s_{i}'} $, etc, no se puede continuar con artimetica clásica y debemos operar con aritmetica modular. Para esto, se ha usado el algoritmo extendido de euclides y la identidad de bezout. 

Para esto se tiene el siguiente código:

\begin{lstlisting}
def boring(args = None):
    pass
\end{lstlisting}


\section{Manual de usuario}

\section{Experimentacion}
\section{Conclusiones finales}
\section{Bibliografia}
\end{document}
