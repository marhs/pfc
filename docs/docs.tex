\documentclass[11pt, spanish]{report}
\usepackage[spanish]{babel}
\selectlanguage{spanish}
\usepackage[utf8]{inputenc}
\usepackage{amsmath}
% TODO Configurar el documento para que use codigo python
\usepackage{listings}

\begin{document}

\title{Descripción y analisis de un protocolo autenticado de transferencia de claves en grupo basado en la comparticion de secretos.}
\author{Marco Herrero Serna,\\
        Escuela Tecnica Superior de Ingeniería Informatica,\\
        Universidad de Sevilla\\
        \texttt{me@marhs.de}}
\date{\today}
\maketitle


% TODO Rellenar el abstract
\begin{abstract}
Un estudio y posterior implementación de un protocolo de transferencia de claves en grupo. Está basado en usuarios autenticados a un Key Generatino Center que las distribuye mediante tecnicas de compartición de secretos. Se estudia tambien las debilidades de este metodo, las mejores que ofrece sobre otros diferetes y el state of the art. 
\end{abstract}

\section{Introducción}
 
Contexto sobre el que trabajamos

\section{Descripcion del problema}
    \subsection{Acuerdo de clave}
    \subsection{Casos prácticos o ejemplos }
    \subsection{Soluciones externas }
    \subsection{Comparativas }

\section{Solucion presentada}
Descripcion y limitaciones

\section{Análisis de requisitos y planificación}
    \subsection{Planificacion }
    Búsqueda de problemas
    Ataque de problemas
    Apartado practico, etc

\section{Implementación}
Interfaz grafica, kivy pyinstaller, etc. 

Uno de los problemas planteados para la implementación es la solución de la ecuación

\begin{equation}
    K' = g^{s_{i}'} * g^{s_{i}+r_{i}} / g^{r_{i}}
\end{equation}

dónde al conocer los valores ya calculados de $ g^{s_{i}'} $, etc, no se puede continuar con artimetica clásica y debemos operar con aritmetica modular. Para esto, se ha usado el algoritmo extendido de euclides y la identidad de bezout. 

Para esto se tiene el siguiente código:

\begin{lstlisting}
def boring(args = None):
    pass
\end{lstlisting}


\section{Manual de usuario}

\section{Experimentacion}
\section{Conclusiones finales}
\section{Bibliografia}
\end{document}
